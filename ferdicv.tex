%%%%%%%%%%%%%%%%%
% This is an example CV created using altacv.cls (v1.0, 16 August 2016) written by
% LianTze Lim (liantze@gmail.com), based on the 
% Cv created by BusinessInsider at http://www.businessinsider.my/a-sample-resume-for-marissa-mayer-2016-7/?r=US&IR=T
% 
%% It may be distributed and/or modified under the
%% conditions of the LaTeX Project Public License, either version 1.3
%% of this license or (at your option) any later version.
%% The latest version of this license is in
%%    http://www.latex-project.org/lppl.txt
%% and version 1.3 or later is part of all distributions of LaTeX
%% version 2003/12/01 or later.
%%%%%%%%%%%%%%%%

\documentclass[10pt,a4paper]{altacv}

%% AltaCV uses the fontawesome and academicon fonts
%% and packages. 
%% See texdoc.net/pkg/fontawecome and http://texdoc.net/pkg/academicons for full list of symbols.
%% 
%% Compile with LuaLaTeX for best results. If you
%% want to use XeLaTeX, you'll need to install
%% Academicons.ttf in your operating system's font %% folder.

% Change the page layout if you need to
\geometry{left=1.5cm,right=9.5cm,marginparwidth=6.8cm,marginparsep=1.2cm,top=1.5cm,bottom=2cm}

% Change the font if you want to.
\setmainfont{Lato}

% Change the colours if you want to
\definecolor{LightPink}{HTML}{F06292}
\definecolor{LightPurple}{HTML}{BA68C8}
\definecolor{LightBlue}{HTML}{3656DA}
\definecolor{AtlanticGreen}{HTML}{2A8E82}

\definecolor{VividPurple}{HTML}{3E0097}
\definecolor{SlateGrey}{HTML}{2E2E2E}
\definecolor{LightGrey}{HTML}{666666}
\colorlet{heading}{AtlanticGreen} %VividPurple
\colorlet{accent}{AtlanticGreen} %VividPurple
\colorlet{emphasis}{VividPurple} %SlateGrey
\colorlet{body}{LightGrey} %LightGrey

% Change the bullets for itemize and rating marker
% for \cvskill if you want to
\renewcommand{\itemmarker}{{\small\textbullet}}
\renewcommand{\ratingmarker}{\faCircle}

%% sample.bib contains your publications
\addbibresource{ferdicv.bib}

\begin{document}
\name{Ferdian Adi Pratama}
\tagline{Robotics Developer}
% Cropped to square from https://en.wikipedia.org/wiki/Marissa_Mayer#/media/File:Marissa_Mayer_May_2014_(cropped).jpg, CC-BY 2.0
\photo{2.5cm}{pp}
\personalinfo{%
  % Not all of these are required!
  % You can add your own with \printinfo{symbol}{detail}
  \email{ferd1@protonmail.com}
  \phone{080-770-15047}
  \birthday{January 9th, 1988}
%  \homepage{ferdianap.github.io}
  \skype{ferdianap}
  \home{Kyodo 1-41-17 room 205, Setagaya, Tokyo} % \mailadress{} for envelope symbol
%  \location{BSD City, Indonesia}
%  \twitter{itsferdee}
%  \linkedin{linkedin.com/in/mylinkedin}
%   \github{github.com/ferdianap} % I'm just making this up though.
%   \orcid{orcid.org/0000-0000-0000-0000} % Obviously making this up too
}

%% Make the header extend all the way to the right, if you want. Extend the right margin by 8cm (=6.8cm marginparwidth + 1.2cm marginparsep)
\begin{adjustwidth}{}{-8cm}
\makecvheader
\end{adjustwidth}

%% Provide the file name containing the sidebar contents as an optional parameter to \cvsection.
%% You can always just use \marginpar{...} if you do
%% not need to align the top of the contents to any
%% \cvsection title in the "main" bar.
\cvsection[fpage1sidebar]{Industry Experience}

\cvevent{Lead R\&D Engineer}{MJI Robotics}{March 2017 -- Present}{Tokyo, JAPAN}
MJI Robotics develops TAPIA, a communication robot. As the lead R\&D engineer, I manage a small research team to provide AI modules for TAPIA with the state-of-the-art technology of vision deep learning, currently focused on object detection and face recognition. I also manage the deployment of the modules and the infrastructure of the production server.

\divider

\cvevent{Software Engineering Intern}{PLENGoer Robotics (now PLEN Robotics)}{Sept 2016 -- Dec 2016}{Osaka, JAPAN}
PLENGoer Robotics is a startup company co-founded by PLEN Project and the GoerTek Group and specializes in the development of practical household and personal service robots.
Specialized in robot architecture and software development, I provide technical advises and support with ROS programming and its application to the PLEN CUBE robot series, including the corresponding Android app.

\divider

\cvevent{Electrical Engineering Intern}{Demag Cranes \& Components, GmbH}{Feb 2009 -- Aug 2009}{Wetter, GERMANY}
Demag Group is one of the world's leading suppliers of industrial cranes, crane components and services of the Demag brand.

I was assigned in Elektrotechnick Handling Technology department during a 6 months internship, and developed infrared circuit modules for overhead crane controller boards

\divider

\cvevent{Mechatronics Engineering Intern}{Siemens Vocational Training Center}{Sept 2007 --  Dec 2007}{Cilegon, INDONESIA}
Handled milling machine, lathe machine, electrical components and circuit, programmed a CNC Lathe machine

\divider

\cvsection{Education}

\cvevent{Ph.D.\ in Information Science}{Japan Advanced Institute of Science and Technology}{2013 -- 2016}{Nomi, JAPAN}
\begin{itemize}
\item Advisor: Nak Young Chong
\item Thesis: Enforcing Personalized Human-Robot Interaction through an Integrated Epigenetic Robot Architecture
%\item My Ph.D. is about the design of developmental robot architecture inspired from psychological studies, where the robot is able to developmentally learn in a symbolic manner based on visual stimuli regarding its interaction with human (verbally) and the environment (physically).
%\item I developed my own image processing module using the currently available technique (saliency detection, object detection and tracking), using ROS and OpenCV.
%\item I have experience with Kinect using PointCloud and multiple grasping point of an object, as well as a stereo-vision and generic RGB camera.
\end{itemize}

\cvsection[fpage2sidebar]{Education}

\cvevent{M.S.\ in Information Science}{Japan Advanced Institute of Science and Technology}{2011 -- 2013}{Nomi, JAPAN}

\divider

\cvevent{B.S.\ in Mechatronics Engineering}{Swiss German University}{2006 -- 2010}{Tangerang, INDONESIA}

\divider

%\cvsection{A Day of My Life}

% Adapted from @Jake's answer from http://tex.stackexchange.com/a/82729/226
% \wheelchart{outer radius}{inner radius}{
% comma-separated list of value/text width/color/detail}
%\wheelchart{1.5cm}{0.5cm}{%
%  10/13em/accent!30/Sleeping \& dreaming about work, 
%  25/9em/accent!60/Public resolving issues with Yahoo!\ investors,
%  5/12em/accent!10/New York \& San Francisco Ballet Jawbone board member, 
%  20/12em/accent!40/Spending time with family,
%  5/8em/accent!20/Business development for Yahoo!\ after the Verizon acquisition,
%  30/9em/accent/Showing Yahoo!\ employees that their work has meaning,
%  5/8em/accent!20/Baking cupcakes
%}

%\clearpage

\cvsection{Research Talks}

\cvevent{\color{accent}ERIS: Epigenetic Robot Intelligent System}{\color{emphasis}IEEE RO-MAN Workshop}{2015}{Kobe, JAPAN}

\divider

\cvevent{\color{accent}Toward Epigenetic Architecture Interconnection: Context-influenced Long-Term Memory}{\color{emphasis}The 1st Hanyang-JAIST Joint Workshop on Decision and Planning}{2014}{Nomi, JAPAN}

\divider

\cvsection{Selected Publications}

\nocite{*}

%\printbibliography[heading=pubtype,title={\printinfo{\faBook}{Books}},type=book]
%
%\divider

\printbibliography[heading=pubtype,title={\printinfo{\faFileTextO}{Journal Articles}}, type=article]

\divider

\printbibliography[heading=pubtype,title={\printinfo{\faGroup}{Conference Proceedings}},type=inproceedings]
%Complete list of publication is available on the website.

\divider

\end{document}
